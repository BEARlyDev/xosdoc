\documentclass[10pt]{report}


\usepackage[pdftex]{graphicx}
\usepackage{url} 
\usepackage[dvips, bookmarks, colorlinks=false, pdfborder={0 0 0}, pdftitle={<pdf title here>}, pdfauthor={<author's name here>}, pdfsubject={<subject here>}, pdfkeywords={<keywords here>}]{hyperref} 
\usepackage[final]{pdfpages}
\usepackage{multirow}
\usepackage[table]{xcolor} 
\usepackage{subfigure}
\usepackage{booktabs}


\newtheorem{example}{Example}[section]
\newtheorem{defn}{Def}
\newcommand{\ESIM}{\textsc{E}\small{\texttt{SIM}}~}
\newcommand\T{\rule{0pt}{3.1ex}}		% To add space b/w words and top \hline
\renewcommand{\figurename}{Fig.}

\newcommand\counter[1]{\arabic{#1} \stepcounter{#1}}
\newcounter{syscall}
\title{XOS \\ eXperimental Operating System \\
Version 1.0}
\author{Dr. K. Muralikrishnan  \\ \texttt{kmurali@nitc.ac.in} \\ {NIT Calicut} }


\begin{document}

\maketitle
\pagebreak

%......................Table of Contents............................%
\thispagestyle{plain}

\tableofcontents
\pagebreak

\chapter{Introduction}
\label{chp:osintro}
\textbf{XOS} (\textit{Experimental Operating System}) is an experimental operating system which is designed to be run on the \textbf{XSM} (\textit{Experimental String Machine}) architecture which is a simulated machine hardware. XOS is intended as an instructional tool to help students learn various aspects about operating systems.
\vspace{0.1in} \\
XOS is programmed using a custom language, \textbf{SPL} (\textit{System Programmer's Language}) which compiles to XSM compatible code. Application programs for XSM are written in \textbf{APL} (\textit{Application Programmer's Language}). 
\vspace{0.1in} \\
The programs, data and operating system code is stored on a disk which has an \textbf{XFS} (\textit{Experimental File System}) in it. \\

The various functionalities of XOS include
\begin{itemize}
\item \textbf{Process Management}, includes scheduling and dispatching processes to the CPU. XOS is capable of \textit{multiprogramming} (the ability to run more than one process.  simultaneously). Refer Chapter~\ref{chp:process}
\item \textbf{Memory Management}, involves allocating memory for processes, demand paging (loading memory pages from the disk as and when required). Refer Chapter~\ref{chp:memory management}
\item \textbf{System Calls}. XOS provides various system calls for the user processes to execute certain kernel level operations. Refer Chapter~\ref{chp:system_calls}
\end{itemize} 


\chapter{Memory Organization}
\label{chp:memoryorg}
\index{Memory Organization}

The operating system organizes memory as given below:
\begin{figure}[htp!] \small
	\centering
	\begin{tabular}{|c|c|c|c|}
	\toprule
		\textbf{Page No.} & \textbf{Contents} & \textbf{Word Address} & \textbf{$\#$ of words}\\
	\toprule
		0   & \hyperref[lbl:romcode]{ROM Code} 		& 0 -- 511 & 512\\ \hline 
		1   & \hyperref[lbl:oscode]{OS Startup code} / Scratchpad* 	& 512 -- 1023 & 512 \\ \hline 
		\multirow{5}{*}{2} 
			& \hyperref[lbl:pgtbl]{Per-Process Page Tables}   & 1024 -- 1279 & 256\\ \cline{2-4} 
			& \hyperref[lbl:memlst]{Memory Free List}  & 1280 -- 1343 & 64  \\ \cline{2-4}
			& \hyperref[lbl:gft]{System-wide Open File Table}  & 1344 -- 1471 & 128 \\ \cline{2-4} 
			& Unallocated & 1472 -- 1535 & 64\\ \hline 
		3 & \multirow{2}{*}
			{\hyperref[lbl:disklst]{Ready List of PCBs}} & \multirow{2}{*}{1536 -- 2559} & \multirow{2}{*}
			{1024}\\ \cline{1-1} 
		4 & 				&  & \\ \hline 
		5 & \hyperref[lbl:fat]{File Allocation Table} & 2560 -- 3071 &  512 \\ 
		\hline
		6 & \hyperref[lbl:fat]{Disk Free List} & 3072 -- 3583 &  512 \\ 
		\hline
		7 -- 8 & \hyperref[lbl:fat]{Exception Handler} & 3584 -- 4607 &  1024 \\ 
		\hline
		9 -- 10 & \hyperref[lbl:fat]{Timer Interrupt Routine} & 4608 -- 5631 &  1024 \\ 
		\hline
		11 -- 12 & \hyperref[lbl:fat]{Interrupt 1 Routine} & 5632 -- 6655 &  1024 \\ 
		\hline
		13 -- 14 & \hyperref[lbl:fat]{Interrupt 2 Routine} & 6656 -- 7679 &  1024 \\ 
		\hline
		15 -- 16 & \hyperref[lbl:fat]{Interrupt 3 Routine} & 7680 -- 8703 &  1024 \\ 
		\hline
		17 -- 18 & \hyperref[lbl:fat]{Interrupt 4 Routine} & 8704 -- 9727 &  1024 \\ 
		\hline
		19 -- 20 & \hyperref[lbl:fat]{Interrupt 5 Routine} & 9728 -- 10751 &  1024 \\ 
		\hline
		21 -- 22 & \hyperref[lbl:fat]{Interrupt 6 Routine} & 10752 -- 11775 &  1024 \\ 
		\hline
		23 -- 24 & \hyperref[lbl:fat]{Interrupt 7 Routine} & 11776 -- 12799 &  1024 \\ 
		\hline
		25 &  &  &  512 $\times$ 48\\ 
		\vdots & INIT and User Programs & 12800 -- 32767 &   19968 \\
		63 &  &  & \\  
		\hline
	\end{tabular}
	\caption{Outline of the main memory}
	*Page Number 1 (OS Startup Code) will be used as scratchpad after bootup
	\index{Memory!Organization}
\end{figure}

\begin{itemize}
	
	\item \textbf{OS Startup code}, loads the INIT process to memory and sets up data structures like FAT, Disk Free List, and Memory Free List. It also loads the Interrupt Routines and Exception handler from the disk to the memory. Refer Section~\ref{sec:os startup code}
	\item \textbf{Per-Process Page Tables}, used for address translation of logical addresses to physical address. Refer Section~\ref{sec:pagetable}
	\item \textbf{Memory Free List}, is a list of free memory locations in the memory. Refer Section~\ref{sec:mem free list}
	\item \textbf{System-wide Open File Table}, contains a details of files which are opened by the processes. Refer Section~\ref{sec:file table}
	\item \textbf{Ready List of PCBs}, is a list of Process Control Blocks, which indicates the ready and terminated processes. Refer Section~\ref{sec:pcb}
	\item \textbf{Memory Copy of File Allocation Table}, contains details about files stored on the disk, Refer Section~\ref{sec:fat}
	\item \textbf{Memory Copy of Disk Free List}, contains details about used and used blocks in the disk, Refer Section~\ref{sec:disk free list}
	\item \textbf{Exception Handler}, contains the kernel code to be executed during various exceptions, Refer Section~\ref{sec:ex_handler}
	\item \textbf{Timer Interrupt Routine}, contains the kernel code to be executed during a timer interrupt. Refer Section~\ref{sec:timer}
	
	\item \textbf{Interrupt Routines}, contains kernel code to be executed during interrupts (1 to 7). Refer Section~\ref{sec:interrupt}
	\item \textbf{INIT and User Programs}, is the memory space allocated for user programs in execution. Refer Section~\ref{sec:init and userprogs}
\end{itemize}




\chapter{Process Management}
\label{chp:process}
\index{Process}

\section{Introduction}
Any program in its execution is called a \textbf{process}. Processes will be loaded into memory before they start their execution. Each process occupies \textbf{at most 4 pages} of the memory. The processor generates logical addresses with respect to a process during execution, which is translated to the physical address. This translation is done by the machine using page tables. \\ 

The XSM architecture supports demand paging and so the machine does not fix the number of processes that can be run on it. However XOS has limited the number of process running simultaneously to 32, due to limitations in number of PCBs in the Ready List (Refer Section~\ref{sec:pcb}) and the number of Per-Process Page Tables (Refer Section~\ref{sec:pagetable})\\




\section{Process Structure}
\label{sec:procstruct}
\index{Process!Structure}
A process in the memory has the following structure.
\begin{itemize}
	\item \textbf{Code Area :} \index{Process!Code Area} These are pages of the memory that contain the executable code loaded from the disk. 
	\item \textbf{Stack :} This is the user stack used for program execution. The variables and data used during execution of program is stored in the stack. It grows in the direction of increasing word address. The location of the stack is fixed at the 4th page of the process.
\end{itemize}

Figure~\ref{fig:process structure} shows the process structure. \\



\begin{figure}[htp!] 
	\centering
	\begin{tabular}{|c|} 
		\textbf{Contents}    \\ \cline{1-1}
		\multirow{2}{*}{Code} \\
				       \\ \cline{1-1}
		\noalign{\smash{\llap{\lower2pt\hbox{\tt BP$\longrightarrow$}}}}
		 \\
	  \\
		\noalign{\smash{\llap{\raise2pt\hbox{\tt $\bigg \downarrow$ }}}}
		Stack  \\ \cline{1-1}
		\noalign{\smash{\llap{\lower2pt\hbox{\tt SP$\longrightarrow$}}}}
	\end{tabular}
	\caption{Logical Address Space of a Process}
	\label{fig:process structure}
\end{figure}




\section{Process Control Block (PCB)}
\label{sec:pcb}
\index{Process Data Structures!Process Control Block}
It contains data pertaining to the current state of the process. The size of the PCB is \textbf{32 words}. Refer figure~\ref{fig:pcb}.\\


	\begin{figure}[htp!]
		\centering
		\begin{tabular}{|c|c|c|c|c|c|c|c|c|c|}
			\hline
			0 & 1 & 2 & 3 & 4 & 5 & 6 & 7 -- 14 & 15 - 30 &  31 \\
			\hline
			PID & STATE & BP & SP & IP & PTBR & PTLR & R0 -- R7 & Per-Process & Free\\ & & & & & & & & Open File Table &   \\
			\hline
		\end{tabular}
		\caption{Structure of Process Control Block}
		\label{fig:pcb}
	\end{figure}

 \subsection{Process Identifier (PID)}
 \label{sec:pid}
 The process identifier is a number from 0 to 31, which identifies the processes in memory. 
 
	 \subsection{Process State} 
	 \label{sec:processstate}
	 The process state corresponding to a process, indicated by \texttt{STATE} in the PCB stores the state of that process in the memory. A process can be in one of the following states. 
	\begin{itemize}
		\item \textbf{0} for \textit{terminated}, i.e. process has completed execution
		\item \textbf{1} for \textit{ready}, i.e. process is waiting for the CPU to start execution.
		\item \textbf{2} for \textit{running}, i.e. the process is currently running in the CPU
	\end{itemize}
	
	  \subsection{Registers}
	  \label{sec:registers} 
\begin{itemize}
	\item \textbf{IP}: The word address of the currently executing instruction is stored in the \texttt{IP} (Instruction Pointer) register. The value of this register cannot be changed explicitly by any instruction. 
	
	\index{Registers!IP}
	\item \textbf{BP}:   The base address of the user stack is stored in the \texttt{BP} (Base Pointer) register. \index{Registers!BP}
	\item \textbf{SP}: The address of the stack top is stored in the \texttt{SP} (Stack Pointer) register. \index{Registers!SP}
	\item \textbf{PTBR}: The physical address of the Per-Process Page Table of the process is stored in the \texttt{PTBR} (Page Table Base Register).\index{Registers!PTBR}
	\item  \textbf{PTLR}: The length of the Per-Process Page Table (No. of entries) is stored in the \texttt{PTLR} (Page Table Length Register). It is fixed as \textbf{4} for every process in XOS. \index{Registers!PTLR}
	
\end{itemize}
Each process has its own set of values for the various registers. Words  7 -- 14 in the PCB stores the values of the registers associated with the process .
	
	
\subsection{Per-Process Open File Table} 
\label{sec:processfile table}
	 
	  The Per-Process Open File Table contains details of files opened by the corresponding process. Every entry in this table occupies 2 words. A maximum of 8 files can be opened by a process at a time, i.e. up to 8 entries in the PCB. It is stored in the PCB from words 15 to 30. Its structure is given below


		\begin{figure}[htp!]
		\centering
		\begin{tabular}{|c|c|}
			\textit{1 word} & \textit{1 word} \\
			\hline
			Pointer to system-wide & \texttt{LSEEK} position\\ open file table entry &  \\
			\hline
		\end{tabular}
		\caption{Structure of Per-Process Open File Table}
		\label{fig:processfile table}
	\end{figure}
	
	For an invalid entry, the value of pointer to system wide open file table is set to -1 .
\begin{itemize}

\item
The OS maintains a system wide open file table which contains details of all the files that are opened by processes (Refer Section~\ref{sec:file table}). The entry in the Per-Process File Table points to the System-wide Open File Table entry corresponding to the file. 
\item It also stores the \texttt{LSEEK} position for the file, which indicates the word in the file to which the process currently points to for read/write operations. 
\end{itemize}




\section{Ready List}
\index{Process!Ready List}
\label{sec:readylist}

The list of PCBs stored in the memory is used as a Ready List by the operating system to schedule processes to CPU. The \texttt{STATE} in the PCB indicates whether a process is ready for execution or not.  A new process in memory is scheduled for execution by circularly traversing through the list of PCBs stored in memory and selecting the first Ready process after the PCB of the currently running process in the list.\\

A maximum of 32 PCBs can be stored in the memory, and hence the maximum number of processes that can be run simultaneously is limited to 32. The PCB list is stored in pages 3 and 4 in the memory (words 1536 -- 2559)

\section{The Per-Process Page Tables}
\label{sec:pagetable}
\index{Process!Page Table}
Every process in XOS has a Per-Process Page Table. A total of 32 PCBs and 32 Page Tables in total are available, which limits the number of processes that can be run to 32. \\

		\begin{figure}[htp!]
		\centering
		\begin{tabular}{|c|c|}
		\hline
		  	 	& 			\\ 
		 	Physical Page Number & Auxiliary Information  \\
			 	& 			\\  \hline
		\end{tabular}
		\caption{Structure of a valid Page Table Entry}
		\label{fig:pagetable}
	\end{figure}


The Per-Process Page Table stores the physical page number corresponding to each logical page associated with the process. The logical page number can vary from 0 to 3 for each process.  Therefore, each process has 4 entries in the page table. Per-Process Page Tables are stored in Page 2, words 1024 -- 1279 in the memory ( 256 words = 32 processes $\times$ 4 entries )\\

When a process is loaded, the actual pages are not loaded into memory. In \textbf{demand paging}, the actual pages are loaded only when the pages are accessed for the first time (Refer Section~\ref{sec:paging}). Once all pages are loaded, the first word of each entry contains the physical page number where the data specified by the logical address resides in the memory.


The second word contains \textbf{auxiliary information}. The first two bits of auxiliary information are reserved as \textit{reference(R) bit} and \textit{valid/invalid(V) bit}. The remaining bits are not used by XOS, but can be used for future enhancements. The details of bits in Auxiliary information is given below. 

\begin{figure}
	\centering
	\begin{tabular}{|c|c|c|c|c|}
		0 & 1 & 2 & ... & 15 \\
		\hline
		R & V & \textbackslash 0 & ... & \textbackslash 0 \\
		\hline 
	\end{tabular}
	\caption{Structure of Auxiliary Information}
\end{figure}


\begin{itemize}
\item
\textbf{Reference Bit (R)}: Initially, this bit is set to 0 (unreferenced) by the machine. On a page access, this bit is set to 1 by the machine. This bit is used for page replacement by the OS. 
\item \textbf{Valid/Invalid Bit (V)} : This bit indicates whether the entry of the page table is valid or invalid. The \textit{Valid/Invalid} bit has value 1 if the first word of this entry corresponds to a valid physical page number. It has value 0 if the entry is invalid. 

The first word of an invalid Per-process page table entry is either -1 (indicates that there is no physical page corresponding to the logical address) or a disk block number (the physical page corresponding to the logical address resides in this disk block and needs to be loaded to memory). The Valid/Invalid bit is set by the OS. If memory access is made to a page whose page table entry is invalid, the machine transfers control to the Exception Handler routine, which is responsible for loading the correct physical page.
\end{itemize}

An example is given below
		\begin{figure}[htp!]
		\centering
		\begin{tabular}{|c|c|}
			\textit{Physical Page Number} & \textit{Auxiliary Information (Reference and Valid Bit)} \\
			\cline{0-1}
			36 & 01  \\
			\cline{0-1}
			311 & 00 \\
			\cline{0-1}
			-1 &  000 \\
			\cline{0-1}
			490 & 00 \\
			\cline{0-1}
		\end{tabular}
		\caption{Structure of Per-Process Page Table}
		\label{fig:processfile table}
	\end{figure}
	
	In the above example :
\begin{itemize}
	\item Reference bit of every entry is set as 0, indicating unreferenced
	\item The 1st entry is a valid page in memory as the valid bit is 1.
	\item The 2nd entry is invalid (valid bit is 0) and the disk block no corresponding to that entry is stored (311).
	\item The 3rd entry is invalid. There is no physical page associated with this logical address.
	\item The 4th entry is invalid and the disk block no stored is 490. This corresponds to a page in the swap area.
\end{itemize}		
	  
	
	
\section{Multiprogramming}
\label{sec:multiprogramming}
	
The operating system allows multiple processes to be run on the machine and manages the system resources among these processes. This process of simultaneous execution of multiple processes is known as \emph{multiprogramming}.  \\

To support multiprogramming in the system, the kernel makes use of the \emph{scheduler} which is present in the Timer Interrupt Service Routine in Pages 9 and 10 of the memory.

\section{INIT and User Processes}
\label{sec:init and userprogs}

The \texttt{INIT} process is the first user program that is loaded by the OS after start up. The \texttt{INIT} and other user processes uses the memory pages 25 - 63 for execution (Code Area and Stack). 





\chapter{Memory Management}
\label{chp:memory management}
\section{Introduction}

XSM uses a paging mechanism for address translation. XOS supports virtual memory, i.e. it supports execution of processes that are not completely in memory. It follows \textit{pure demand paging} strategy for memory management. Pages are allocated as and when required during execution. 

\section{Paging}
\label{sec:paging}
\index{Page }

Paging is the memory management scheme that permits the physical address space of a process to be non-contiguous. Each process has its own page table (Refer Section~\ref{sec:pagetable}), which is used for paging. 

 The Per-Process Page Table contains information relating to the actual location in the memory. Each valid entry of a page table contains the page number in the memory where the data specified by the logical address resides. The address of Page Table of the currently executing process is stored in \texttt{PTBR} and length of the page table is set to 4 in \texttt{PTLR} of the machine. 
 

	
\section{Memory Free List}
\label{sec:mem free list}
\index{Memory!Free list}

The free list of the memory consists of 64 entries. Each entry is of size one word. Thus, the total size of the free list is thus 64 words. It is present in words 1280 to 1343 in memory. (words 256 to 319 of Page ) of the memory. Refer Chapter~\ref{chp:memoryorg}. Each entry of the free list contains a value of either 0 or 1 indicating whether the corresponding page in the memory is free or not respectively. When a page is shared by more than one process, the entry stores the number of processes that share the page.

 
\section{Virtual Memory}
\label{sec:virtualmem}

XOS allows virtual memory management, i.e. running processes without having all the pages in memory. It makes use of a backing store or \textbf{\textit{swap}} in the disk to replace pages from the memory and allocate the emptied memory to another process. This increases the total number of processes that can be run simultaneously on the OS.\\

When a process starts executing, no memory pages are allocated for it. Initially its Per-process page tables are set with the block numbers of the disk blocks which contain the data blocks of the program. For each page table entry, the \textit{Auxiliary Information} are intialized to 0 (invalid) and 0 (unreferenced). When a page is referenced for the first time, it triggers a page fault exception (since valid bit is set as 0). The \textit{Exception Handler Routine} is responsible for loading the required page from the disk to the memory. This strategy of loading pages when accessed for the first time, is known as \textbf{Pure Demand Paging}.\\

On encountering a page fault exception, the Exception Handler Routine loads the required page from the disk to a free page in the memory. If no free page is available in the memory, a page replacement technique is used to select a victim page. The page replacement technique used in XOS is a \textit{second chance algorithm} (Refer Silberschatz, Galvin, Gagne: Operating System Concepts) which uses the reference bits in the auxiliary information. The victim page is swapped out to to the disk (swap area) to accommodate the required page.




 \chapter{Files}
\label{chp:files}
The operating system requires accessing the file system (XFS) while loading programs, and reading data from the files. The operating system maintains a memory copy of the file system data structures like FAT(File Allocation Table) and Disk Free List (Refer Chapter~\ref{chp:memoryorg}). It is loaded from the disk to the memory during operating system boot.\\

 Apart from the file system data structures XOS maintains details about files opened by all processes in the System-wide Open File Table. XOS uses a \texttt{scratchpad} to access files in the memory which will be explained further in this chapter.
 

\section{File Allocation Table (FAT)}
\index{File Allocation Table}
\label{sec:fat}
\emph{File allocation table} (FAT) is a table that has an entry for each file present in the disk. FAT is stored in page number 5 in the memory. \\

	\index{File Allocation Table!Location in disk}


The structure of a FAT entry is shown below 

\begin{figure}[htp!] \small
	\centering
	\begin{tabular}{|c|c|c|c|}
		\hline
		0 & 1 & 2 & 3 -- 7 \\
		\hline
		File Name & File Size & Block \# of basic block & \dots{} Unused \dots \\
		\hline
	\end{tabular}
	\caption{Structure of a FAT entry}
	\label{fig:fat_entry}
\end{figure}



\section{Disk Free List}
\index{Disk Free List}
\label{sec:disk free list}
The Disk Free List is a data structure used for keeping track of unused blocks in the disk. The memory copy of Disk Free List is stored in the \textit{page number} 6. It is stored in \textit{block number} 20 in the disk. 


\section{System Wide Open File Table}
\label{sec:file table}
	
This data structure maintains details about all open files in the system. It is located from words 1344 to 1471 of the memory (in Page 2). System Wide Open File Table consists of a maximum of 64 entries. 
Therefore, there can be at most 64 open files in the system at any time. Each entry of the System Wide Open File Table occupies 2 words. It has the following structure as shown in figure~\ref{fig:file table}.

	 \begin{figure}[h!]
		 \centering
			\begin{tabular}{|c|c|}
				1 word & 1 word	\\		
				\hline
				FAT Index & File Open Count\\
				\hline
			\end{tabular}
		 \caption{Structure of an entry}
		 \label{fig:file table}
	 \end{figure}

	 \begin{itemize}
		 \item \textbf{FAT index :} \index{File Allocation Table!Memory copy} It stores the index of the corresponding file in the FAT. An invalid entry is denoted by -1.

		 \item \textbf{File Open Count :} File Open Count is the number of open instances of the file. When this becomes zero, the entry for the file is invalidated in the System Wide Open File Table.
	 \end{itemize}

The Per-Process Open File Table in the PCB of each process stores information about files opened by the corresponding process. Each entry in the Per-Process Open File Table has the index to the file’s entry in the System-wide Open File Table.

\section{Scratchpad}
\label{sec:scratchpad}
\index{Scratchpad}
There is a specific page of the memory which is reserved to store temporary data. This page is known as the \textit{Scratchpad}. The scratchpad is required since any block of the disk cannot be accessed directly  by a process. It has to be present in the memory for access. Hence, any disk block that has to be read or written into is first brought into the scratchpad. It is then read or modified and written back into the disk. \\

The \textit{page number} 1 of the memory (Refer Chapter \ref{chp:memoryorg}) is used as the scratchpad. Once the OS has booted up there is no need for the OS startup code. So this page can be reused as the scratchpad.


\chapter{System Calls}
\label{chp:system_calls}
\index{System Calls}

\section{Introduction}
System calls are interfaces through which a process communicates with the OS. Each system call has a unique name associated with it (Open, Read, Fork etc). Each of these names maps to a unique system call number. Each system call in turn causes a software interrupt to occur. Note that multiple system calls can be mapped to the same interrupt.

All the arguments to the system call are pushed into the user stack of the process which invokes the system call. The system call number is pushed as the last argument. 


\section{File System Calls}
\label{sec:fssyscall}
\textit{File system calls} are used by a process when it has to create, delete or manipulate \textit{Data files} that reside on the disk(file system). There are seven file system calls. An interrupt is associated with each system call. All the necessary arguments for a system call are available in the user stack with the system call number as the last argument.\\

\subsection{Create}
\label{sec:create()}

APL Syntax : \texttt{int Create(fileName)} \\
System Call No. : 1 \\

This system call is used to create a new file in the file system whose name is specified in the argument. The return value of the \texttt{Create()} system call is 0 if it is a success, and -1 otherwise. If the file already exists, the system call returns 0 (success). It invokes Interrupt 1 Routine.\\


\subsection{Open}
\label{sec:open()}

APL Syntax : \texttt{int Open(fileName)} \\
System Call No. : 2 \\

This system call is used to open an existing file whose name is specified in the argument.  It calls Interrupt 2 Routine. The return value of the \texttt{Create()} system call is an integer value called \texttt{FileDescriptor}, which is the index of the corresponding file's entry in the Per-Process Open File Table.\\


\subsection{Close}
\label{sec:close()}

APL Syntax : \texttt{int Close(fileName)} \\
System Call No. : 3 \\

This system call is used to close an open file.  \texttt{FileDescriptor} is an integer value returned by the corresponding \texttt{Open()} system call. The return value of the \texttt{Close()} system call is 0 if it is a success, and -1 otherwise. It invokes Interrupt 2 Routine. \\


\subsection{Delete}
\label{sec:delete()}

APL Syntax : \texttt{int Delete(fileDescriptor)} \\
System Call No. : 4\\

This system call is used to delete the file from the file system whose name is specified in the argument. The return value of the \texttt{Delete()} system call is 0 if it is a success, and -1 otherwise. It invokes Interrupt 1 Routine. \\


\subsection{Write}
\label{sec:write()}

APL Syntax : \texttt{int Write(fileDescriptor, wordToWrite)}  \\
System Call No. : 5 \\

This system call is used to write one word at the current seek position, into an open file ( identified by \texttt{fileDescriptor} ) from a string/integer variable ( identified by \texttt{wordToWrite} ). The return value of the \texttt{Write()} system call is 0 if it is a success or -1 otherwise. It invokes Interrupt 4 Routine.\\


\subsection{Seek}
\label{sec:seek()}

APL Syntax : \texttt{int Seek(FileDescriptor, newLseek)}  \\
System Call No. : 6 \\

This system call is used to change the current value of the seek position in the per-process open file table entry of a file to the \texttt{newLseek} value. The return value of the \texttt{Seek()} system call is 0 if it is a success, and -1 otherwise. It invokes Interrupt 3 Routine.\\


\subsection{Read}
\label{sec:read()}

APL Syntax : \texttt{int Read(fileDescriptor, wordRead)}  \\
System Call No. : 7 \\

This system call is used to read one word at the current seek position, from an open file ( identified by \texttt{fileDescriptor} ) and store the word to a string/integer variable ( identified by \texttt{wordRead} ). The return value of the \texttt{Read()} system call is 0 if it is a success or -1 otherwise. It invokes Interrupt 3 Routine.\\


	
\section{Process System Calls}
\label{sec:procsyscall}
\index{Process System Calls}
\textit{Process system calls} are used by a process when it has to duplicate itself, execute a new process in its place or when it has to terminate itself. There are three process system calls. An interrupt is associated with each system call. All the necessary arguments for a system call are available in the user stack with the system call number as the last argument.\\


\subsection{Fork}
\label{sec:fork()}

APL Syntax :  \texttt{int Fork()} \\
System Call No. : 8 \\

This system call is used to replicate the process which invoked it. The new process which is created is known as the \emph{child} and the process which invoked this system call is known as its \emph{parent}. The return value of the \texttt{Fork()} system call to the parent process is the PID (\textit{process identifier}) of the child process and -2 for the child process. It invokes Interrupt 5 Routine\\



\subsection{Exec}
\label{sec:exec()}

APL Syntax :  \texttt{int Exec(filename)} \\
System Call No. : 9 \\

This system call is used to load the program, whose name is specified in the argument, in the memory space of the current process and start its execution. The return value of the \texttt{Exec()} system call is -1 if it failed. It invokes Interrupt 6 Routine.\\


\subsection{Exit}
\label{sec:exit()}

APL Syntax :  \texttt{Exit()} \\
System Call No. : 10 \\

This system call is used to terminate the execution of the process which invoked it and removes it from the memory . It schedules the next ready process and starts executing it. When there is no other ready process to run, it halts the machine. It invokes Interrupt 7 Routine. \\



\chapter{System Routines}

\label{chp:sys routines}

The Operating System apart from its various data structures and interfaces it provides to the user processes, has certain routines to execute while start up and during interrupts. These routines are included as the Operating System Routines.

\section{OS Startup Code}
\label{sec:os startup code}
The OS Startup Code resides in the page 1 in the memory. When the machine boots up, the ROM Code loads the OS Startup Code from block 0 in the disk to page 1 in the memory. The OS Startup code initializes all data structures required for the OS, loads the FAT and Disk Free List from file system into the memory and starts execution of the \texttt{INIT} process.



\section{Exception Handler}
\label{sec:ex_handler}
When the machine encounters an exception it sets EFR (Exception Flag Register) with details corresponding to the exception and calls the exception handler routine (pages 7 and 8 in memory).


		\begin{figure}[htp!]
		\centering
		\begin{tabular}{|c|c|c|c|}
		\hline
		\texttt{Value of IP} & \texttt{BadVAddr} & \texttt{Cause} &  \textbackslash 0 \\
		\hline
		\end{tabular}
		\caption{Structure of EFR }
		\end{figure}
		
XOS handles all  exceptions other than \textit{Page Fault} by killing the process which caused the exception. 

\subsubsection{Page Fault Exceptions}
\label{sec:page fault}
The \texttt{Cause} field of \texttt{EFR} for Page Fault Exceptions is \textbf{0}. The logical page which caused the exception to occur (indicated by \texttt{BadVAddr} field in EFR ) will not have a corresponding valid entry in the page table of the process. If the page table entry contains a disk block number, the block is loaded from the disk to a free memory page, and this memory page number is stored in the page table entry . The Valid/Invalid bit is set to 1, and the exception handler returns back to the process.

 

\section{Timer Interrupt Routine}
\label{sec:timer}
The Timer Interrupt Routine is responsible for context switch, i.e. storing the state (values of the registers) of the currently executing process to the PCB, and setting the registers with values from the PCB of the next ready process in the Ready List of PCBs. A scheduler is responsible for selecting a ready process from this list. The Scheduler code is also contained in the Timer Interrupt Routine. The Timer Interrupt routine resides in pages 9 and 10 of the memory.



\section{Interrupt Routines}
\label{sec:interrupt}
The Interrupts from 1 to 7 are invoked by the user processes through system calls. Each interrupt routine has code corresponding to one or more system calls. Every interrupt routine occupies 2 pages in memory. Interrupt routines for interrupts 1 to 7 reside in memory pages 11 to 24.

\end{document}
